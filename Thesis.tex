\documentclass{amsart}

\usepackage{amsmath,amsfonts,amsthm,amssymb,stmaryrd,paralist,tikz,amsthm}
\usepackage[mathscr]{euscript}
\usetikzlibrary{matrix,arrows,decorations.pathmorphing}



\newcommand{\R}{\mathbb{R}}
\newcommand{\Q}{\mathbb{Q}}
\newcommand{\Z}{\mathbb{Z}}
\newcommand{\C}{\mathbb{C}}
\newcommand{\N}{\mathbb{N}}
\newcommand{\D}{\mathbb{D}}
\newcommand{\T}{\mathbb{T}}
\newcommand{\RP}{\mathbb{R}\mathrm{P}}
\newcommand{\CP}{\mathbb{C}\mathrm{P}}
\renewcommand{\H}{\mathbb{H}}
\let\oldS\S
\renewcommand{\S}{\mathbb{S}}
\newcommand{\s}{\mathbb{S}}


\newtheorem{thm}{Theorem}[section]
\newtheorem*{thm*}{Theorem}
\newtheorem{lem}[thm]{Lemma}
\newtheorem*{lem*}{Lemma}
\newtheorem{cor}[thm]{Corollary}
\newtheorem*{cor*}{Corollary}
\newtheorem{prop}[thm]{Proposition}
\newtheorem*{prop*}{Proposition}
\newtheorem{defn}{Definition}
\newtheorem*{defn*}{Definition}
\newtheorem{question}{Question}
\newtheorem*{question*}{Question}


\newcommand{\re}{\mathrm{Re}}
\newcommand{\im}{\mathrm{Im}}
\newcommand{\iprod}[1]{\left< #1 \right>}
\newcommand{\abs}[1]{\left| #1 \right|}
\newcommand{\set}[1]{\left\{ #1 \right\} }
\newcommand{\norm}[1]{ \left\| #1 \right\| }
\newcommand{\del}{\nabla}
\newcommand{\two}{I\!\!I}


\usepackage{fancyhdr}
 \pagestyle{fancy}
 \chead{\sf DRAFT VERSION 1 --- {\today}}
 \setlength{\headheight}{14pt}
 \setlength{\footskip}{0.5in}
 \lhead{} \rhead{} \lfoot{} \cfoot{\thepage} \rfoot{}
 \renewcommand{\headrulewidth}{0pt}
 \renewcommand{\footrulewidth}{0pt}



\usepackage{color}
\definecolor{verydarkblue}{rgb}{0,0,0.4}
\usepackage{hyperref}
\hypersetup{
pdfauthor={Keaton Quinn},
pdftitle={Asymptotically Poincar\'e surfaces in quasi-Fuchsian manifolds},
colorlinks=true,linkcolor=verydarkblue,
citecolor=verydarkblue,urlcolor=verydarkblue
}



\begin{document}

\title{Amazing Thesis}

\author{Keaton Quinn}

\maketitle

\begin{center} {\sf DRAFT VERSION 1 --- {\today} } \end{center}



\section{Introduction}

%%%%%%%%%%%%%%%%%%%%
\section{Preliminaries}

\subsection{The Geometry of Surfaces}

Let $S$ be a closed, oriented, smooth surface. 
A Riemannian metric $g$ on $S$ is a smooth section of the symmetric tensor product of the cotangent bundle of $S$, i.e., $g \in \Gamma(\Sigma^2T^*S)$, such that $g(p): T_pS \times T_pS \to \R$ is an inner product.
The non-degeneracy of $g$ at each point allows us to identity $TS$ with $T^*S$ via the map that sends $v \in T_pS$ to $g_p(\cdot  , v) \in T_p^*S$ and this extends to an identification of vector fields and 1-forms. 
The Levi-Civita connection $\nabla$ of $g$ on $S$ is the unique torsion-free, metric connection on $S$ 
\[
T^\nabla = 0 \text{ and } \nabla g = 0.
\]

Suppose $(M,\tilde{g})$ is a 3-dimensional Riemannian manifold. 
When $f: S \to M$ is an immersion, the pullback tensor $I = f^*\tilde{g}$ is a Riemannian metric on $S$ that we call the First Fundamental Form. 
If we regard $f$ as an identification of $S$ with its image in $M$, then we may also identify $T_pS$ with its image in $T_{f(p)}M$.
We therefore have a $\tilde{g}$-orthogonal splitting
\[
T_pM = T_pS \oplus N_pS.
\]
Here, $N_pS$ is the normal space to the surface in $M$ given by all vectors orthogonal to $S$ at $p$.
The disjoint union of the normal spaces forms a vector bundle (in this case a line bundle) called the Normal Bundle of $S$.
By our assumptions, there is a unit normal vector field $n$ so that $S$ is co-oriented in $M$.

Let $\widetilde{\nabla}$ be the Levi-Civita connection of $\tilde{g}$ on $M$.
By extending vector fields $X$ and $Y$ on $S$ to a neighborhood of $S$ in $M$, we may take $\widetilde{\nabla}_X Y$.
This resulting vector field is not necessarily also tangent to $S$ and so we may decompose it into its tangential part $(\widetilde{\nabla}_X Y)^\top$ and its normal part $(\widetilde{\nabla}_X Y)^\perp$.
The Gauss Formula tells us $(\widetilde{\nabla}_X Y)^\top = \nabla_X Y$ and since the normal bundle is spanned by $n$ we have $(\widetilde{\nabla}_X Y)^\perp = \two(X,Y)n$ for a symmetric 2-tensor field $\two \in \Gamma(\Sigma^2T^*S)$.
This $\two$ we call the Second Fundamental Form.

 
Given a torsion-free connection $\nabla$ on $S$ and a $q$-form $\omega$ with values on $TS$, the exterior covariant derivative of $\omega$ is 
\[
d^\nabla \omega = \mathrm{Alt}(\nabla \omega).
\]
By considering vector fields as 0-forms with values in $TS$ we define the Riemann Curvature Endomorphism of $\nabla$ as the 2-form with values in $\mathrm{End}(TS)$ given by  
\[
R^{\nabla}(X,Y)Z 
= (d^\nabla \circ d^\nabla Z)(X,Y)
= \nabla_X \nabla_Y Z - \nabla_Y \nabla_X Z - \nabla_{[X,Y]}Z.
\]
The Riemann Curvature Tensor is then 
\[
Rm(X,Y,Z,W) = g(R^{\nabla}(X,Y)Z,W).
\]
and the Gaussian curvature of $S$ at a point $p$ is the function $K(g)$ defined by 
\[
K(g)(p) = \frac{Rm(v,w,w,v)}{\|v\|^2\|w\|^2 - g(v,w)}
\]
where $v,w$ is any basis for $T_pS$.
One dimension upwards in $M$, the Sectional Curvature of $\tilde{g}$ is a function on 2-planes in the tangent bundle $sec: \mathrm{Gr}_2(TM) \to \R$ whose value on $\Pi \leq T_pM$ is the Gaussian curvature at $p$ of the image of $\Pi$ in $M$ under the exponential map.

The Gauss Equation relates the Gaussian curvature $K$ of $S$ to the sectional curvature of the ambient manifold $M$ via the first and second fundamental forms.
Since $\two$ is symmetric and since $I$ is non-degenerate, we may form the shape operator $I^{-1}\two: TM \to TM$.
The Gauss Equations states
\[
K(I) = sec(TS) + \det(I^{-1}\two).
\]








%%%%%%%%%%%%%%%%%%%%
\section{Asymptotically Poincar\'e Families}

%%%%%%%%%%%%%%%%%%%%
\section{$k$-surfaces}

%%%%%%%%%%%%%%%%%%%%
\section{Constant Mean Curvature Surfaces}

%%%%%%%%%%%%%%%%%%%%
\section{Future Directions}


































\end{document}