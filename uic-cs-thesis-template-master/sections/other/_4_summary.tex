\summary

We study the asymptotic behavior of certain foliations of ends of quasi-Fuchsian manifolds. 
We introduce a class of such foliations, which we call asymptotically Poincar\'e families as they are asymptotic to a family of surfaces determined naturally by the Poincar\'e metric on the Riemann surface at infinity.
We prove the limiting behavior of any asymptotically Poincar\'e family is completely determined by the geometry of the quasi-Fuchsian manifold $M$ and the conformal structure on the ideal boundary of $M$. 
Specifically, the conformal classes of the first and second fundamental forms $[I_\epsilon]$ and $[\two_\epsilon]$ of the surfaces in an asymptotically Poincar\'e family converge in Teichm\"uller space as $\epsilon \to 0$ to the conformal structure of the surface at infinity of the end of $M$.
At $\epsilon  = 0$, the tangent vectors satisfy $\dot{[I_\epsilon]} = c \mathrm{Re}(\phi)$ and $\dot{[\two_\epsilon]} = 0$ where $\phi$ is the holomorphic quadratic differential of the complex projective structure at infinity and $c$ a constant that depends on the foliation. 

As an application of these results we establish a conjecture of Labourie regarding the asymptotic behavior of the $k$-surface foliation he constructs in \cite{labourie1992}.
We also show that the constant mean curvature foliation of Mazzeo and Pacard \cite{mazzeo-pacard2011} forms an asymptotically Poincar\'e family and so we can describe its asymptotics as well.
