\section{A conjecture of Labourie}
\label{labourie-conjecture-section}

Labourie proved in \cite{labourie1991} that geometrically finite ends of hyperbolic 3-manifolds admit foliations by surfaces of constant Gaussian curvature. 
These he called $k$-surfaces and for each $k \in (-1,0)$ there exists a unique surface in the foliation whose curvature is identically $k$.
Both ends of a quasi-Fuchsian manifold are geometrically finite, hence this foliation result applies in the quasi-Fuchsian case.

Focusing on one end of the quasi-Fuchsian manifold with surface at infinity $X$, in \cite{labourie1992} Labourie discusses how the conformal classes $[I_k]$ and $[\two_k]$ of the first and second fundamental forms of the $k$-surfaces behave as paths in the Teichm\"uller space of $X$. 
When $k \to -1$, he proves that $[\two_k]$ approaches the point at infinity of $\mathcal{T}(X)$ corresponding to the measured geodesic lamination on the convex core of the quasi-Fuchsian manifold. 
When $k \to 0$, he showed both $[I_k]$ and $[\two_k]$ converge to the same point: the complex structure $X$ on the surface at infinity; or in metric terms, converge to the class of the hyperbolic metric $[h]$.
He conjectures that the tangent vector to these paths is related to the holomorphic quadratic differential at infinity $\phi$. 

We establish his conjecture later in this chapter (Theorem \ref{labourie-conjecture-proof}). 
Indeed, Labourie's $k$-surfaces form an asymptotically Poincar\'e family (as we will show). 
Therefore, our main result Theorem \ref{main-result} applies, describing the asymptotic behavior of the $k$-surfaces as $k \to 0$. 
In the process, our work gives an alternative proof to Labourie's theorem on the existence of $k$-surfaces (at least for $k$ near 0), in this case for quasi-Fuchsian manifolds: see Theorem \ref{k-surfaces-existence}.
