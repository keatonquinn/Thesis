\section{Geometry of Epstein surfaces}
\label{epstein-geometry}

The first fundamental form of the Epstein surface for the metric $\sigma$ is given by $I(\sigma) = \mathrm{Ep}_\sigma^*(g_{\H^3})$ for $g_{\H^3}$ the metric of $\H^3$. 
It is given by 
\[
I(\sigma) = 4|B(g_{\CP^1},\sigma)|^2\sigma^{-1} + \frac{1}{4}(1-K(\sigma))^2\sigma + 2(1-K(\sigma))\text{Re}(B(g_{\CP^1},\sigma)).
\]
The second fundamental form (relative to the normal lift $\widehat{\mathrm{Ep}}_\sigma$) is 
\[
\two(\sigma)
= 4|B(g_{\CP^1},\sigma)|^2\sigma^{-1} - \frac{1}{4} (1 - K(\sigma)^2)\sigma - 2 K(\sigma) \text{Re}(B(g_{\CP^1},\sigma)).
\]
These formulas are derived in \cite[Eqns.~3.2-3.3]{dumas2017}.
Here $K(\sigma)$ is the Gaussian curvature of $\sigma$ and $B(g_{\CP^1},\sigma)$ the Schwarzian derivative of $\sigma$ with respect to a M\"obius flat metric. 
We note that when the Epstein surface is embedded, its second fundamental form is negative definite due to our convenient choice of normal vector field.
Writing $\det(g)$ for the determinant of the matrix representation of $g$ relative to some local frame for the tangent bundle, the metrics on $\sigma$ and $I(\sigma)$ on $S$ have determinants related by
\[
\det(I(\sigma)) = \frac{1}{16}\left( (1-K(\sigma))^2 - 16 |B(g_{\CP^1},\sigma)|^2\sigma^{-2} \right)^2 \det(\sigma).
\]
This equation is independent of the frame and has in intrinsic meaning, namely it describes the ratio of volume forms of these two metrics. 
We can compute the Gaussian curvature by $K(I(\sigma)) = -1 + \det(I(\sigma)^{-1}\two(\sigma))$ and the mean curvature by $H(\mathrm{Ep}_\sigma) = \frac{1}{2}\mathrm{tr}(I(\sigma)^{-1}\two(\sigma))$. 
We obtain
\[
K(I(\sigma))
= \frac{4K(\sigma)}{(1-K(\sigma))^2 - 16|B(g_{\CP^1},\sigma)|^2\sigma^{-2}}
\]
and
\[
H(\mathrm{Ep}_\sigma)
= \frac{K(\sigma)^2 - 1 - 16 |B(g_{\CP^1},\sigma)|^2\sigma^{-2}}{(K(\sigma) - 1)^2 - 16 |B(g_{\CP^1},\sigma)|^2\sigma^{-2}}.
\]

 
In the quasi-Fuchsian setting, if $\sigma$ is a $\Gamma$-invariant conformal metric on $\Omega$ then each term in the above equations is also $\Gamma$-invariant. 
This may be less clear for the quadratic differential $B(g_{\CP^1},\sigma)$ since the M\"obius flat metric $g_{\CP^1}$ is not itself $\Gamma$-invariant. 
However, we see that for $\gamma \in \Gamma$ we have $\gamma^*B(g_{\CP^1},\sigma) = B(\gamma^*g_{\CP^1},\gamma^*\sigma) = B(\gamma^* g_{\CP^1},\sigma)$, by naturality of Schwarzian derivatives of conformal metrics. 
The metric $\gamma^* g_{\CP^1}$ is still a M\"obius flat metric, and so $B(\gamma^* g_{\CP^1},\sigma) = B(g_{\CP^1},\sigma)$, implying $B(g_{\CP^1}, \sigma)$ is $\Gamma$-invariant. 
Therefore, $B(g_{\CP^1},\sigma)$ induces a quadratic differential on $X$, which we will denote by $B(\sigma)$.

In summary of the above, we have the following Gaussian and mean curvatures of the Epstein surfaces in $M$.
\begin{lem}
\label{curvature-epstein}
The Gaussian curvature for the Epstein surface $\mathrm{Ep}_\sigma : X \to M$ is given by
\[
K(I(\sigma))
= \frac{4K(\sigma)}{(1-K(\sigma))^2 - 16 |B(\sigma)|^2\sigma^{-2}},
\]
and the mean curvature by 
\[
\pushQED{\qed}
H(\mathrm{Ep}_\sigma)
= \frac{K(\sigma)^2 - 1 - 16 |B(\sigma)|^2\sigma^{-2}}{(K(\sigma) - 1)^2 - 16 |B(\sigma)|^2\sigma^{-2}}.
\qedhere
\popQED
\]
\end{lem}
These are now equations on the compact Riemann surface $X$.