\section{The Epstein map}


As an inverse to the visual metric construction above, the Epstein map takes a domain $\Omega$ in $\CP^1$ and a conformal metric $\sigma$ on $\Omega$ and describes a strictly convex surface $f: \Omega \to \H^3$. 
This surface has the property that $V_{f(z)}(z) = \sigma(z)$ and this property can be used to derive a formula for the surface. 
Specifically, expanding upon this last condition and using the affine parameter discussed above, we see that 
\[
\sigma(z) = V_{f(z)}(z) = e^{2 [f(z),z]} \overset{\circ}{\sigma}(z),
\]
or that $f(z)$ lies on the horosphere based at $z$ with affine parameter
\[
[f(z),z] = \frac{1}{2} \log \left( \frac{\sigma(z)}{\overset{\circ}{\sigma}(z)} \right) =: \rho(z).
\]
Hence, we know which horosphere based at $z$ that $f(z)$ must lie on. 

There is a convenient choice of normal vector field. 
At $f(z)$, the geodesic in the direction $n(f(z))$ must end at $z$ in order for $f$ to be an inverse to the visual metric construction. 
The normal vectors to a horosphere pointing to its base have this property, so we define $n(f(z))$ to be the normal vector to the horosphere based at $z$ with affine parameter $\rho(z)$. 
Since $n$ must be normal to the image surface $S = f(\Omega)$, this identifies the tangent spaces to the sought-after $S$ with the tangent spaces to the horospheres. 
And this identifies the surface $S$ with the envelope of the family of horospheres
\[
\mathcal{H}(\Omega,\sigma) = \{ \text{Horosphere based at $z$ with parameter $\rho(z)$} \ | \ z \in \Omega \}.
\]


In \cite{epstein1984}, Epstein derives an equation for such an envelope. 
Working in the ball model and taking $z$ both as a point in $\CP^1 \cong S^2$ as well as a unit vector in $\R^3$, he shows the desired map is 
\[
\mathrm{Ep}_\sigma(z) = \frac{|\overset{\circ}{\nabla}\rho(z)|^2 + e^{2\rho(z)} - 1}{|\overset{\circ}{\nabla}\rho(z)|^2 + (e^{\rho(z)}+1)^2} z + \frac{2}{|\overset{\circ}{\nabla}\rho(z)|^2 + (e^{\rho(z)}+1)^2} \overset{\circ}{\nabla}\rho(z),
\]
where $\overset{\circ}{\nabla}\rho$ is the gradient of $\rho$ with respect to the spherical metric on $S^2$. 
This construction leads to the following theorem. 


\begin{thm}[Epstein \cite{epstein1984}]
Let $\Omega$ be a domain in $\CP^1$  and $\sigma$ a $C^k$ conformal metric on $\Omega$, then there exists a unique $C^{k-1}$ map $\mathrm{Ep}_\sigma : \Omega \to \H^3$, called the Epstein map of $\Omega$ for the metric $\sigma$, such that for all $z \in \Omega$,
\[
V_{\mathrm{Ep}_\sigma(z)}(z) = \sigma(z).
\]
Moreover, the image of a point $z$ depends only on the 1-jet of $\sigma$ at $z$.
\label{epstein-map-def}
\end{thm}

Epstein's original construction uses the ball model of hyperbolic space to define the Epstein map. 
In \cite{dumas2017}, Dumas gives a model independent definition of the map using an $\text{SL}_2\C$-frame field. 
It proceeds as follows. 
Choose an affine coordinate chart $z$ on $\CP^1$ that distinguishes a point $0 \in \Omega$ and $\infty \notin \Omega$. 
Then, on the geodesic in $\H^3$ with ideal endpoints $0$ and $\infty$, there exists a unique point $p$ such that the visual metric from $p$ at $0$ is the Euclidean metric of this affine chart, $V_p(0) = |dz|^2$. 
The Epstein map is an $\mathrm{SL}_2\C$-frame orbit of this point.     


\begin{prop}[\cite{dumas2017}]
On a domain $\Omega$ in $\CP^1$ write $\sigma = e^{2\eta}|dz|^2$. Define the $\mathrm{SL}_2,\C$-frame field $\widetilde{\mathrm{Ep}}_\sigma: \Omega \to \mathrm{SL}_2\C$ by 
\[
\widetilde{\mathrm{Ep}}_\sigma(z) =
\begin{pmatrix}
1 & z \\
0 & 1
\end{pmatrix}
\begin{pmatrix}
1 & 0 \\
\eta_z & 1
\end{pmatrix}
\begin{pmatrix}
e^{-\eta/2} & 0 \\
0 & e^{\eta/2}
\end{pmatrix},
\]
then the Epstein map is given by 
\[
\mathrm{Ep}_\sigma(z) = \widetilde{\mathrm{Ep}}_\sigma(z) \cdot p.
\]
\end{prop}

 

Even though we call the image an Epstein surface, the Epstein map need not be an immersion. 
Indeed, if $\sigma$ is itself a visual metric then the Epstein map for $\sigma$ is constant. 
However, the lift of $\mathrm{Ep}_\sigma$ from $\Omega$ to the unit tangent bundle of hyperbolic space given by 
\[
\widehat{\mathrm{Ep}}_\sigma(z) = (\text{Ep}_\sigma(z),z) 
\] 
is an immersion (here we are using the trivialization $U\H^3 \cong \H^3 \times \CP^1$ defined above). 
This lift can be thought of as providing a unit ``normal'' vector field for the Epstein surface even when the Epstein map is not an immersion. 
Indeed, this lift agrees with a unit normal vector field when the surface is immersed and so we will simply refer to it as the normal field from now on. 

Because the Epstein map is unique, it is natural with respect to the action of $\mathrm{SL}_2\C$ in the following sense. 
Suppose $M \in \mathrm{SL}_2\C$, then the following diagram commutes:
\[
\begin{tikzpicture}[scale=0.9]
\node (1) at (0,2) {$(\Omega,\sigma)$};
\node (2) at (3,2) {$(M(\Omega),M_*\sigma)$};
\node (3) at (0,0) {$\H^3$};
\node (4) at (3,0) {$\H^3$};


\draw[->] (1) to node [above] {$M$} (2);
\draw[->] (1) to node [left] {$\mathrm{Ep}$} (3);
\draw[->] (3) to node [above] {$M$} (4);
\draw[->] (2) to node [right] {$\mathrm{Ep}$} (4);
\end{tikzpicture}
\]
That is, $\mathrm{Ep}_{M_*\sigma}(M(z)) = M( \mathrm{Ep}_{\sigma}(z))$. 
This is because both $M \circ \mathrm{Ep}_\sigma$ and $\mathrm{Ep}_{M_*\sigma} \circ M$ as maps on $\Omega$ satisfy the visual metric condition from Theorem \ref{epstein-map-def} (see \cite{anderson1998} for more details ).

This allows us to define Epstein maps on certain quotients. 
Suppose in general that $\Gamma$ is a subgroup of $\mathrm{SL}_2\C$ acting freely and properly discontinuously on $\H^3 \cup \CP^1$ leaving a domain $\Omega$ invariant. 
Then $\Omega/\Gamma$ inherits a Riemann surface structure. 
Call this structure $X$ and let $\sigma$ be a conformal metric  on $X$.
Lift this to $\tilde{\sigma}$ on $\Omega$, which is $\Gamma$-invariant. 
Then $\mathrm{Ep}_{\tilde{\sigma}}: \Omega \to \H^3$ is $\Gamma$-equivariant and therefore descends to a map $\mathrm{Ep}_\sigma : X \to \H^3/ \Gamma$. 
In particular, when $\Gamma$ is a quasi-Fuchsian group and $\Omega$ a component of the domain of discontinuity, each conformal metric $\sigma$ on the surface at infinity $X$ gives rise to a map from $X$ into the quasi-Fuchsian manifold $M$.

Uniqueness also shows us that the surfaces parallel to an Epstein surface are themselves Epstein surfaces. 
More specifically, let $g^t : U \H^3 \to \H^3$ denote the time-$t$ geodesic flow projected down to $\H^3$.
Thus for a unit tangent vector $v$ on $\H^3$ we have $g^t(v) = \exp_p(tv)$.
Using the lift of an Epstein surface to $U\H^3$ described above, each Epstein surface gives rise to a family of surfaces by applying the geodesic flow (and projecting to $\H^3$). 
That is, we have the flowed surfaces $g^t \circ \widehat{\mathrm{Ep}}_\sigma(\Omega)$. 
These surfaces are themselves Epstein surfaces corresponding to scalar multiples of $\sigma$. 
Indeed, since the parallel flow of a horosphere is a horosphere, we know $[g^t(\widehat{\mathrm{Ep}}_\sigma(z)),z] = [\mathrm{Ep}_\sigma(z),z] + t$. 
This shows us
\[
V_{g^t(\widehat{\mathrm{Ep}}_\sigma(z))}(z) = e^{2[g^t(\widehat{\mathrm{Ep}}_\sigma(z)),z]}\overset{\circ}{\sigma}(z) = e^{2t}e^{2[\mathrm{Ep}_\sigma(z),z]}\overset{\circ}{\sigma}(z) = e^{2t}\sigma(z).
\]
But the unique map that satisfies this equality is $\mathrm{Ep}_{e^{2t}\sigma}$. 
In summary, we have the following lemma, attributed to Thurston (unpublished work) by Epstein in \cite{epstein1984}.
\begin{lem}
\label{epstein-flow}
Let $\Omega$ be a domain in $\CP^1$ and $\sigma$ a conformal metric on $\Omega$.
Then 
\[
g^t \circ \widehat{\mathrm{Ep}}_\sigma  = \mathrm{Ep}_{e^{2t} \sigma}.
\]
That is, flowing the Epstein surface for $\sigma$ for time $t$ in the normal direction corresponds to taking the Epstein surface for the metric $e^{2t}\sigma$.
\end{lem}
