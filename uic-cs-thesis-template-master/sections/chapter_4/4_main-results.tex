\section{Main results}





So, we work in the Sobolev setting of functions $H^s(X)$ and tensors $\mathrm{Met}^s(X)$ and $\mathrm{Conf}^s(X)$.
Since $K$ and $B$ are smooth functions of $\sigma$ and its derivatives we have that they both extend to functions on Sobolev classes of metrics.
Hence if $\sigma \in \mathrm{Conf}^s(X)$ then $I(\sigma)$ and $-\two(\sigma)$ belong to $\mathrm{Met}^{s-2}(X)$.
We will obtain results with these extensions and then argue our results are independent of the chosen $s$. 



\begin{prop}
\label{thm-in-sobolev}
Suppose $S_\epsilon$ is an asymptotically Poincar\'e family of surfaces. 
Let $\gamma : [0,1) \to \mathrm{Met}^s(X)$ be the extension of $f\sigma$ thought of as taking values in a class of Sobolev metrics for a fixed $s > 3$. 
Then the first and second fundamental forms $I \circ \sigma: (0,1) \to \mathrm{Met}^{s-2}(X)$ and  $\two \circ \sigma: (0,1) \to \mathrm{Met}^{s-2}(X)$ satisfy
\[
I_\epsilon := 4 f'(0) \epsilon I(\sigma(\epsilon)) \to h
\quad \text{ and } \quad
\two_\epsilon :=  - 4 f'(0) \epsilon \two(\sigma(\epsilon)) \to h
\quad \text{ as } \epsilon \to 0.
\]
Moreover, $I_\epsilon$ and $\two_\epsilon$ are differentiable at $\epsilon = 0$ and their tangent vectors are given by 
\[
\dot{I_\epsilon} = \dot{\gamma} + \left(2 f'(0) - \frac{1}{2}\frac{f''(0)}{f'(0)} \right) h + 4 f'(0) \mathrm{Re}(\phi)
\quad \text{ and } \quad
\dot{\two_\epsilon} = \dot{\gamma} - \frac{1}{2}\frac{f''(0)}{f'(0)}h.
\]
\end{prop}

\begin{proof}
We have that $\gamma:[0,1) \to \mathrm{Met}^\infty(X)$ is continuous and differentiable at $\epsilon = 0$. 
Therefore, as $\mathrm{Met}^\infty(X) = \cap_{s > 3} \mathrm{Met}^s(X)$ we also have that $\gamma$ is continuous to $\mathrm{Met}^s$ and differentiable at $\epsilon = 0$. 

Then, we have $I(\sigma(\epsilon)) = I(\frac{1}{f(\epsilon)} \gamma(\epsilon))$ is equal to 
\[
4 f(\epsilon) \frac{|B(\gamma(\epsilon))|^2}{\gamma(\epsilon)} + \frac{1}{4 f(\epsilon)}(1 - f(\epsilon) K(\gamma(\epsilon)))^2 \gamma(\epsilon) + 2(1 - f(\epsilon)K(\gamma(\epsilon)))\mathrm{Re}(B(\gamma(\epsilon))),
\]
which is a smooth tensor independent of $s$. 
Since $f:[0,1) \to \R$ and $\gamma:[0,1) \to \mathrm{Met}^s(X)$ are differentiable at 0 and since $K: \mathrm{Met}^s(X) \to H^{s-2}(X)$ and $B: \mathrm{Conf}^s(X) \to \Gamma^{s-2}(\Sigma^{2}(X))$ are differentiable at the hyperbolic metric $h$, we can write 
\begin{align*}
f(\epsilon) = \epsilon f'(0) + \frac{1}{2}\epsilon^2 f''(0) + O(\epsilon^3), \quad
K(\gamma(\epsilon)) = -1 + O(\epsilon), \quad 
B(\gamma(\epsilon)) = \frac{1}{2} \phi + O(\epsilon).
\end{align*}

Substitution and some simplification gives 
\[
I(\sigma(\epsilon)) = \frac{1}{4\epsilon f'(0)} h + \frac{1}{4 f'(0)} \dot{\gamma} + \left(\frac{1}{2} - \frac{1}{8}\frac{f''(0)}{f'
(0)^2}\right) h + \mathrm{Re}(\phi) + O(\epsilon).
\]
Consequently, 
\[
I_\epsilon =  h + \epsilon \left(\dot{\gamma} + \left(2 f'(0) - \frac{1}{2}\frac{f''(0)}{f'(0)} \right) h + 4 f'(0) \mathrm{Re}(\phi) \right) + O(\epsilon^2).
\]
The same reasoning will give 
\[
\two_\epsilon = h + \epsilon \left( \dot{\gamma} - \frac{1}{2}\frac{f''(0)}{f'(0)} h \right) + O(\epsilon^2).
\]
The result then follows.
\end{proof}


We now prove our main result, Theorem \ref{main-thm-intro} from the Introduction. 
Recall that $[\two]$ denotes the point in Teichm\"uller space corresponding to the conformal class of $-\two$.


\begin{thm}\label{main-result}
Let $S_\epsilon$ for $\epsilon \in (0,1)$ be an asymptotically Poincar\'e family of surfaces with metrics at infinity $\sigma(\epsilon)$. 
If $h$ is the hyperbolic metric of $X$ and $\phi$  the holomorphic quadratic differential at infinity, then in Teichm\"uller space $\mathcal{T}(X)$ we have 
\[
[I(\sigma(\epsilon))] \to [h]
\quad \text{ and } \quad
[\two(\sigma(\epsilon))] \to [h]
\quad \text{ as } \epsilon \to 0.
\]
Moreover, the tangent vectors in $T_{[h]} \mathcal{T}(X)$ are given by 
\[
\dot{[I(\sigma(\epsilon))]}  = 4 f'(0) \mathrm{Re}(\phi) \quad \text{ and } \quad \dot{[\two(\sigma(\epsilon))]} = 0.
\]
\end{thm}

\begin{proof}
Using the notation from  Proposition \ref{thm-in-sobolev}, for each $s > 3$ we have that $I_\epsilon$ and $\two_\epsilon$, which are paths through smooth tensors, converge in $\mathrm{Met}^s(X)$ to the hyperbolic metric $h$. 
Since $\mathrm{Met}^\infty(X)  = \cap \mathrm{Met}^s(X)$ we know that $I_\epsilon$ and $\two_\epsilon$ converge to $h$ in $\mathrm{Met}^\infty(X)$. 
Moreover, since the projection $\mathrm{Met}^\infty(X) \to \mathcal{T}(X)$ is continuous, and since $[I(\sigma(\epsilon))] = [I_\epsilon]$ and $[\two(\sigma(\epsilon))] = [\two_\epsilon]$, we have that
\[
[I(\sigma(\epsilon))] \to [h]
\quad \text{ and } \quad 
[\two(\sigma(\epsilon))] \to [h]
\quad \text{ as } \epsilon \to 0.
\]
Since both paths converge to $[h]$ at $\epsilon = 0$ we can extend them to continuous paths $[0,1) \to \mathcal{T}(S)$. 
Since $[I(\sigma(\epsilon))]$ and  $[\two(\sigma(\epsilon))]$ agree with $[I_\epsilon]$ and $[\two_\epsilon]$, respectively, we also have they are differentiable at $\epsilon = 0$.

From Proposition \ref{thm-in-sobolev} we know that 
\[
\dot{I_\epsilon}  = \dot{\gamma} + \left(2 f'(0) - \frac{1}{2}\frac{f''(0)}{f'(0)} \right) h + 4 f'(0) \mathrm{Re}(\phi)
\quad \text{ and }\quad 
\dot{\two_\epsilon} = \dot{\gamma} - \frac{1}{2}\frac{f''(0)}{f'(0)} h.
\]
Since $\phi$ is holomorphic, $4 f'(0) \mathrm{Re}(\phi)$ is trace-free and divergence-free. 
On the other hand, $(2 f'(0) - (1/2)f''(0)/f'(0)) h$ is pure trace and so belongs to the $\mathrm{Diff}_0^s(X) \rtimes P^s(X)$ orbit of $h$. Furthermore, for any $s$ it belongs to the group orbit of $h$. 
Hence, the derivative of the projection at $h$ removes this term.
Similarly, $\gamma$ is a path in the group orbit of $h$ and so $\dot{\gamma}$ projects to 0. 
We then have
\begin{align*}
\dot{[I(\sigma(\epsilon))]}
= \dot{[I_\epsilon]}
= d \pi_h \left(\dot{\gamma} + (2 f'(0) - (1/2)f''(0)/f'(0)) h + 4 f'(0) \mathrm{Re}(\phi) \right) 
= 4 f'(0) \mathrm{Re}(\phi)
\end{align*}
and
\[
\dot{[\two(\sigma(\epsilon))]} = \dot{[\two_\epsilon]} = d \pi_h (\dot{\gamma}- (1/2)f''(0)/f'(0)) h) = 0,
\]
as claimed.
\end{proof}
