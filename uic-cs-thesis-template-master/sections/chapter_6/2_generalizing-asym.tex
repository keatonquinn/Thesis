\section{Generalizing asymptotically Poincar\'e families}

Asymptotically Poincar\'e families are families of Epstein surfaces for conformal metrics $\sigma(\epsilon)$ such that $f(\epsilon)\sigma(\epsilon) \to h$ as $\epsilon \to 0$.
What happens when $h$ is replaced with some other conformal metric $\sigma_0$?
This is, suppose we consider families of Epstein surfaces for the conformal metrics $\sigma(\epsilon)$ such that $f(\epsilon)\sigma(\epsilon) \to \sigma_0$ in $\mathrm{Met}^\infty(S)$ as $\epsilon \to 0$.
We may ask
\begin{itemize}
\item Do the first and second fundamental forms still converge to the conformal class at infinity? 
\item What are the tangent vectors at $\epsilon = 0$, and are they related to the Schwarzian derivative of $\sigma_0$? 
\end{itemize}

Preliminary calculations suggest that yes, $[I(\sigma(\epsilon))]$ and $[\two(\sigma(\epsilon))]$ converge to $[\sigma_0] = [h]$.
Moreover, we expect that it is still the case that $\dot{[\two(\sigma(\epsilon))]} = 0$.
It also appears that $\dot{[I(\sigma(\epsilon))]}$ is indeed related to $\mathrm{Re}(B(\sigma_0))$. 
However, $B(\sigma_0)$ need not be holomorphic, i.e., if $\sigma_0$ does not have constant curvature, so the tangent vector should be the real part of the holomorphic part of $B(\sigma_0)$. 
It would be nice to find a more explicit description of the holomorphic part of a quadratic differential, and to verify these preliminary computations.  

